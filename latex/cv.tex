\documentclass[paper=a4,fontsize=11pt]{scrartcl} % KOMA-article class
\usepackage{luacode}
\usepackage[left=1.5cm, right=1.5cm, top=1.5cm, bottom=1.5cm]{geometry}
\usepackage{multicol}
\usepackage[english]{babel}
\usepackage{helvet}
\usepackage[utf8x]{inputenc}
\usepackage[protrusion=true,expansion=true]{microtype}
\usepackage{amsmath,amsfonts,amsthm}     % Math packages
\usepackage{graphicx}                    % Enable pdflatex
\usepackage{xcolor}            % Colors by their 'svgnames'
\usepackage{geometry}
	\textheight=700px                    % Saving trees ;-)
\usepackage{url}
\usepackage[T1]{fontenc}
\usepackage[scaled=0.85]{beramono}
\usepackage{scrextend}
\usepackage{xifthen}
\usepackage{etoolbox}
\usepackage{tikz}
	\usetikzlibrary{calc}

\usepackage{xstring}


\raggedbottom
\frenchspacing              % Better looking spacings after periods
\pagestyle{empty}           % No pagenumbers/headers/footers

%%% Custom sectioning (sectsty package)
%%% ------------------------------------------------------------
\usepackage{sectsty}

\sectionfont{%			            % Change font of \section command
	\usefont{OT1}{phv}{b}{n}%		% bch-b-n: CharterBT-Bold font
	\sectionrule{0pt}{0pt}{-5pt}{3pt}}

%%% Macros
%%% ------------------------------------------------------------
\definecolor{PARTCOLOR}{RGB}{24, 58, 145}
\definecolor{DESCOLOR}{RGB}{68, 68, 68}

\graphicspath{ {../img/} }
\newlength{\spacebox}
\settowidth{\spacebox}{8888888888}			% Box to align text
\newcommand{\sepspace}{\vspace*{1em}}		% Vertical space macro
\newcommand{\MyName}[1]{ % Name
		\noindent
		\Huge \usefont{OT1}{phv}{b}{n} #1
		\par \normalsize \normalfont}		
\newcommand{\MySlogan}[1]{ % Slogan (optional)
		\noindent
		\large \textit{#1}
		\normalsize \normalfont}
\newcommand{\NewPart}[1]{\section*{\uppercase{\textcolor{PARTCOLOR}{#1}}}}
\newcommand{\PersonalEntry}[3][0.5em]{
		\noindent
		\textbf{\textit{#2}}% Entry name (birth, address, etc.)
		\strut\hspace{#1}\large#3\normalsize     % Entry value
}
%\ExperienceEntry{company}{location}{title}{start}{end}{desc}
\newcommand{\ExperienceEntry}[6]{
	\noindent
	\textbf{#1}, #2 -- \textit{#3} \\
	\textsc{#4 -- #5} \\
	\textcolor{DESCOLOR}{#6}
	\sepspace
}
%\EducationEntry{school}{location}{degree}{start}{end}{grade}
\newcommand{\EducationEntry}[7]{
	\noindent
	\textbf{#1}, #2 -- \textit{#3}\\
	\textsc{#4 -- #5} \\
	\textcolor{DESCOLOR}{\textsc{#6}}\\
	\begin{addmargin}[1em]{2em}
	#7
	\end{addmargin}
}
%\EducationEntry{school}{location}{degree}{start}{end}{grade}
\newcommand{\ProjectEntry}[3]{
	\noindent
	\textbf{#1} -- \textit{#2} \\
	\textcolor{DESCOLOR}{#3}
	\sepspace
}
%\AddressEntry{flat}{building}{city,country}{postcode}
\newcommand{\AddressEntry}[2]{
	\PersonalEntry{location}{\texttt{#1, #2}}
	
}
%\SkillEntry{skill}{desc}
\newcommand{\SkillEntry}[2]{
	\noindent
	\textbf{#1}
	\begin{multicols}{3}
	\noindent
	\textcolor{DESCOLOR}{\textsc{#2}}
	\end{multicols}
	\sepspace
}

\newcommand{\placepicture}[4][center]{%
  % [#1]: box anchor: center (default) | 
  %                 south west | west | north west | north |
  %                 north east | east | south east | south | 
  %                 mid west | mid | mid east |
  %                 base west | base | base east 
  % #2: horizontal position (fraction of page width)
  % #3: vertical position (fraction of page height)
  % #4: image name
  %
  \tikz[remember picture,overlay,x=\paperwidth,y=\paperheight]{%
    \node[anchor=#1,inner sep=0pt]
    at ($(current page.south west)+(#2,#3)$) {\includegraphics[width=0.2\textwidth]{#4}};
  }%
}
%\Header{name}{email}{phone}{city}{country}{motivation}{[picture]}
\newcommand{\Header}[8]{
	\MyName{#1}
	\vskip 15pt
	\PersonalEntry{github}{​\texttt{github.com/#2}}\\
	\PersonalEntry{e-mail}{\texttt{#3}}\\
	\PersonalEntry{phone}{​\texttt{#4}}\\
	\AddressEntry{#5}{#6}

	\edef\mytemp{{#8}}%
	\expandafter\ifstrequal\mytemp{none}{}{\placepicture[north east]{0.9}{0.98}{#8}}
	\vskip 15pt
	\MySlogan{#7}
}

\luadirect{
package.path = './?.lua;' .. package.path
util = require("getskill.lua")}
 
\renewcommand{\baselinestretch}{0.99} 
\renewcommand{\familydefault}{\sfdefault}
%%% Begin Document
%%% ------------------------------------------------------------
\begin{document}
	\begin{luacode}
	require("lualibs.lua")

	function getjsonfile (file)
	    local f, s
	      f = io.open(file, 'r')
	        s = f:read('*a')
	        f.close()
	        return s
	 end

	paramtable =  utilities.json.tolua(getjsonfile('../latex/aux/temp.json'))
    \end{luacode}

\Header{\luadirect{
	   local name = paramtable["name"]
	   tex.print(name)}}
	   {\luadirect{
	   local github = paramtable["github"]
	   tex.print(github)}}
	   {\luadirect{
	   local email = paramtable["email"]
	   tex.print(email)}}
	   {\luadirect{
	   local phone = paramtable["phone"]
	   tex.print(phone)}}
	   {\luadirect{
	   local city = paramtable["location_city"]
	   tex.print(city)}}
	   {\luadirect{
	   local country = paramtable["location_country"]
	   tex.print(country)}}
	   {Recent engineering graduate with experience and interest in both software and hardware design. Currently looking for a software engineering role that would allow me to expand upon and improve the skills I acquired at University.}
	   {\luadirect{
	   local picture = paramtable["cv_image"]
	   tex.print(picture)}}

\NewPart{Work Experience}
\ExperienceEntry{Kelvin Nanotechnology}{Glasgow, United Kingdom}{MEng intern}{jul 2018}{dec 2018}{Researched an experimental method of substrate cleaning in the University of Glasgow's Nanofabrication centre. The study involved analysing large amounts of image data, for which a bespoke commandline tool was developed.}

\ExperienceEntry{Thom Micro Systems}{Falkirk, United Kingdom}{App developer}{jun 2017}{aug 2017}{Summer​ ​ internship​ ​in​ ​which​ ​a ​ ​ prototype​ ​ contract costing​ ​ application was created​.​ ​ The​ ​ purpose​ ​ of​ ​ the​ ​ placement​ ​ was​ ​ to
determine​ ​ whether​ ​ the​ ​ Xamarin​ ​ Forms​ ​ IDE​ ​ was​ ​ a ​ ​ suitable
tool​ ​ to​ ​ develop​ ​ a ​ ​ multi-platform​ ​ application​ ​ with​ ​ a ​ ​ single
codebase.​ ​ Produced​ ​ working​ ​ Android​ ​ and​ ​ Universal
Windows​ ​ applications.}

\ExperienceEntry{Subcity Radio}{Glasgow, United Kingdom}{Volunteer web developer}{oct 2016}{apr 2018}{Worked sporadically during studies to create ​ a ​ ​ visual​ ​ sitemap​ ​ of​ ​ the​ ​ Subcity​ ​ website​ ​ for
archival​ ​ purposes.​ Performed various maintenance duties in moving the website's codebase from PHP to Python.}

\NewPart{Education}
\EducationEntry{University of Glasgow}{United Kingdom}{MEng Electronics and Software Engineering}{sep 2014}{jun 2019}{degree - 2:1}
{\noindent \textbf{Subjects included:}\\ Algorithms and Data Structures, Functional Programming, Web App Development, Realtime Embedded Systems, Control Systems, Digital Electronics, Analogue Electronics.\par
\vskip 10pt
\noindent \textbf{Dissertation: "The Efficiency of CO$_2$ Snow Jet Cleaning on Semiconductor Wafers"}\\ Developed a commandline Java application for processing microscope imaging data. The application counts the number of particles in an image and displays them sorted by area. It was distributed to potential users on Node Package Manager.\par
\vskip 10pt
\noindent \textbf{Additional information:}\\ Participated in a charity hackathon hosted by J.P. Morgan (Code for Good). Worked with Django in a team to develop a web application for a non-profit organization.\par}

\vskip 35pt

\EducationEntry{Arab Unity School}{United Arab Emirates}{A--levels}{sep 2012}{jun 2014}{mathematics, physics, chemistry -- A, A, A}{ }

\NewPart{Projects}
% \ProjectEntry{n--body​}{Personal project}{A project that implements a quad-tree and the Barnes-Hut algorithm to approximate the gravitational forces between bodies in a system.}
\ProjectEntry{Raycasting}{Personal project}{A simulation of a light source interacting with opaque walls. Written in Javascript, using the p5.js library and hosted on github.}

\ProjectEntry{Beacon​ Tracking​ GUI​}{Team project}{A​ ​ third-year​ ​ university​ ​ project​ ​ in​ ​ which​ ​ the team ​ worked​ ​ closely
with​ ​ an​ ​ external​ ​ client​ ​ to​ ​ provide​ ​ a ​ ​ graphical​ ​ way​ ​ to​ ​ display and​ ​ update​ ​ the​ ​ position​ ​ of​ ​ multiple​ ​ tracking​ ​ devices​ ​ on​ ​ a floorplan. Written in Javascript using the Meteor.js framework, and using d3.js for data visualisation.}

\ProjectEntry{Electronic​ ​ Design​ ​ Project}{Team project}{A​ ​ second-year​ ​ university​ ​ project​ ​ to​ ​ design,​ ​ prototype​ ​ and​ ​ build​ ​ a heart-rate​ ​ monitor​ ​ using​ ​ a ​ ​microcontroller​ ​ and​ ​ analogue​ ​ signal processing.}

\ProjectEntry{Game​ ​ of​ ​ Life​ ​ Visualiser}{Personal project}{A​ ​ simple​ ​ program​ ​ that​ ​ displays​ ​ the​ ​ evolution​ ​ of​ ​ patterns​ ​ in Conway’s​ ​ Game​ ​ of​ ​ Life. Written in Java and hosted on github.}

\ProjectEntry{Ultrasound communications}{Team project}{Experimented with underwater power transmission and communication using ultrasound. Involved the use of a full bridge rectifier, a power management IC, and programming of an Arduino microcontroller using C.}

\NewPart{Skills}
\luadirect{
local skills = paramtable["checklists"]
printSkills(skills)}
% 

% \SkillEntry{Software packages / Tools}{eclipse​ ​ ide\\​orcad​ pspice\\​orcad ​pcb ​editor\\​xamarin forms\\​android​ studio\\​git\\​p5.js\\d3.js\\jquery\\imagej api\\npm\\numpy\\\LaTeX}
\end{document}

% \SkillEntry{Programming languages}{java \\​ ​ python \\​ ​ html \\​ ​ css \\​ ​ javascript \\​ ​ c \\​ c++ \\ ​c\# \\​ sql \\​ ​ vhdl \\ xaml \\​ ​ vb \\ ​ haskell \\ uml}

% \SkillEntry{Software packages / Tools}{eclipse​ ​ ide \\​ ​ orcad​ pspice \\​ ​ orcad ​ pcb ​ editor \\​ ​ xamarin forms \\​ ​ android​ ​ studio \\​ ​ git \\​ ​ bitbucket \\​ ​ gitlab \\ p5.js \\ d3.js \\ node.js \\ jquery \\ imagej api \\ npm \\ numpy \\ \LaTeX \\ autodesk EAGLE \\ selenium }

% local skills = paramtable["checklists"]
% for skillCount = 1, #skills do
% 	local skill = skills[skillCount]
% 	local skillMembers = skill[2]
% 	local texString = ""
% 	for skillMemberCount = 1, #skillMembers do
% 		local member = skillMembers[skillMemberCount]
% 		if member[2] == 1 then
% 			texString = texString .. member[1]
% 		end
% 	end
% 	tex.print("\SkillEntry{skill}{texString}")
% end